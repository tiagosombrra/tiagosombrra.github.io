\chapter{Introdução}
\label{cap:introducao}

%Para começar a usar este \textit{template}, na plataforma \textit{ShareLatex}, vá nas opções (três barras vermelhas horizontais) no canto esquerdo superior da tela e clique em "Copiar Projeto" e dê um novo nome para o projeto. 

Para começar a utilizar este \textit{template}, siga o tutorial clicando no seguinte \textit{link}:
\url{http://www.biblioteca.ufc.br/images/arquivos/instrucoes_modelos/tutorial_sharelatex.pdf}

Neste \textit{template}, o autor irá encontrar diversas instruções e exemplos dos recursos do uso do \LaTeX~na plataforma \textit{ShareLatex}. O \LaTeX~foi desenvolvido, inicialmente, na década de 80, por Leslie Lamport e é utilizado amplamente na produção de textos matemáticos e científicos, devido a sua alta qualidade tipográfica \cite{goossens1994latex}. 

O \textit{ShareLatex} é uma plataforma \textit{online} que pode ser acessado por meio de qualquer navegador de internet até mesmo de um \textit{smartphone}. Essa plataforma dispensa a instalação de aplicativos no computador para desenvolver trabalhos em \LaTeX. Também, não é necessário instalar \textit{packages}, ou seja, pacotes que permitem diferentes efeitos na formatação e no visual do trabalho. Todos os \textit{packages} que este \textit{template} utiliza são encontrados \textit{online}. 

Apresentam-se, também, neste modelo, algumas orientações de como desenvolver um trabalho acadêmico. Entretanto, este arquivo deve ser editado pelo autor de acordo com o seu trabalho sendo que a formatação já está de acordo com o aceito pela Universidade Federal do Ceará.  

A introdução, tem como finalidade, dar ao leitor uma visão concisa do tema investigado, ressaltando-se o assunto de forma delimitada, ou seja, enquadrando-o sob a perspectiva de uma área do conhecimento, de forma que fique evidente sobre o que se está investigando; a justificativa da escolha do tema; os objetivos do trabalho; o objeto de pesquisa que será investigado. Observe que não se divide a introdução em seções, mas a mesma informa como o trabalho ao todo está organizado.



%Testando o símbolo $\symE$

%\lipsum[5]  % Simulador de texto, ou seja, é um gerador de lero-lero.

%	\begin{alineas}
%		\item Lorem ipsum dolor sit amet, consectetur adipiscing elit. Nunc dictum sed tortor nec viverra.
%		\item Praesent vitae nulla varius, pulvinar quam at, dapibus nisi. Aenean in commodo tellus. Mauris molestie est sed justo malesuada, quis feugiat tellus venenatis.
%		\item Praesent quis erat eleifend, lacinia turpis in, tristique tellus. Nunc dictum sed tortor nec viverra.
%		\item Mauris facilisis odio eu ornare tempor. Nunc dictum sed tortor nec viverra.
%		\item Curabitur convallis odio at eros consequat pretium.
%	\end{alineas}
	

	
